\chapter{系统测试}

\section{测试环境和总体方案}

测试方案:


\subsection{系统功能性测试}
  系统的手机端并没有完全完成, 所以功能性测试主要针对于 Web管理端。 功能分为:

  \begin{itemize}
  \item 学校的创建,修改 (删除功能暂时没有实现)
  \item 班级的创建,删除,修改
  \item 网盘,相册操作
  \item 教师,学生的邀请, 转班, 删除
  \item 帮手的邀请,修改,删除
  \end{itemize}

  \textbf{测试环境为}: Mac osx 10.10, Chrome 39

\subsection{系统界面测试}

  分为PC端测试, 和手机端测试

  \textbf{PC端测试环境}:  Mac osx 10.10, Chrome 39

  \textbf{Mobile端测试环境}: iphone5s








\section{系统功能测试}

\subsection{学校的创建,修改}

\begin{table}[H]
  \centering
  \caption{学校的创建}
  \label{tab:1}
  \begin{tabular}{cc}
    \toprule
    测试名称 & 学校的创建  \\

    %%%%%%%%%%%%%%%%%%%%%%%%%%%%%%%%%
    测试人 & 张鑫语 \\
    测试日期 & 2015-05-11 \\

    \midrule
    测试步骤 		& 点击左边Panel的加号, 输入学校数据 \\
    \midrule
    输入数据 		& 学校名: 测试学校, 学校介绍: 测试学校介绍 \\
    预期结果 		& 测试学校被创建 \\
    实际结果             & 与预期结果完全一致 \\
    \bottomrule
  \end{tabular}
\end{table}


\begin{table}[H]
  \centering
  \caption{学校的修改}
  \label{tab:2}
  \begin{tabular}{cc}
    \toprule
    测试名称 & 学校的修改  \\

    %%%%%%%%%%%%%%%%%%%%%%%%%%%%%%%%%
    测试人 & 张鑫语 \\
    测试日期 & 2015-05-11 \\

    \midrule
    测试步骤 		& 点击相关学校, 之后点击详细信息 \\
    \midrule
    输入数据 		&  修改学校名为: 测试学校修改, 学校介绍: 测试学校介绍 \\
    预期结果 		& 测试学校被修改 \\
    实际结果             & 与预期结果完全一致 \\
    \bottomrule
  \end{tabular}
\end{table}


\subsection{班级的创建,修改}

\begin{table}[H]
  \centering
  \caption{班级的创建}
  \label{tab:3}
  \begin{tabular}{cc}
    \toprule
    测试名称 & 班级的创建  \\

    %%%%%%%%%%%%%%%%%%%%%%%%%%%%%%%%%
    测试人 & 张鑫语 \\
    测试日期 & 2015-05-11 \\

    \midrule
    \multirow{2}{*}{测试步骤}
             & 点击测试学校,之后点击右边Panel上的加号 \\
             & 在出现的创建Panel中输入班级数据 \\

    \midrule
    输入数据 		& 班级名: 测试班级, 班级介绍: 测试班级介绍 \\
    预期结果 		& 测试班级被创建 \\
    实际结果             & 与预期结果完全一致 \\
    \bottomrule
  \end{tabular}
\end{table}

\begin{table}[H]
  \centering
  \caption{班级的修改}
  \label{tab:4}
  \begin{tabular}{cc}
    \toprule
    测试名称 & 班级的创建  \\

    %%%%%%%%%%%%%%%%%%%%%%%%%%%%%%%%%
    测试人 & 张鑫语 \\
    测试日期 & 2015-05-11 \\

    \midrule
    \multirow{2}{*}{测试步骤} 		& 点击测试学校,之后点击右边Panel上的加号 \\
    & 在出现的创建Panel中输入班级数据 \\

    \midrule
    输入数据 		& 班级名: 测试班级, 班级介绍: 测试班级介绍 \\
    预期结果 		& 测试班级被创建 \\
    实际结果             & 与预期结果完全一致 \\
    \bottomrule
  \end{tabular}
\end{table}


\subsection{网盘,相册操作}

\begin{table}[H]
  \centering
  \caption{网盘操作}
  \label{tab:5}
  \begin{tabular}{cc}
    \toprule
    测试名称 & 网盘操作  \\

    %%%%%%%%%%%%%%%%%%%%%%%%%%%%%%%%%
    测试人 & 张鑫语 \\
    测试日期 & 2015-05-11 \\

    \midrule
    \multirow{3}{*}{测试步骤} 		& 点击测试学校, 在右边Panel选择文件管理, 之后点击根文件夹 \\
             & 在右下边Panel点击新建, 新建成功后, 修改文件信息并提交 \\
             & 之后点击删除(都在右下部Panel) \\

    \midrule
    输入数据 		& 任意文件 \\
    预期结果 		& 可以在网盘内新建,修改,删除文件 \\
    实际结果             & 与预期结果完全一致 \\
    \bottomrule
  \end{tabular}
\end{table}


\begin{table}[H]
  \centering
  \caption{相册操作}
  \label{tab:6}
  \begin{tabular}{cc}
    \toprule
    测试名称 & 相册操作  \\

    %%%%%%%%%%%%%%%%%%%%%%%%%%%%%%%%%
    测试人 & 张鑫语 \\
    测试日期 & 2015-05-11 \\

    \midrule
    \multirow{3}{*}{测试步骤} 		& 点击测试学校, 在右边Panel选择相册, 在右下边Panel点击上传 \\
             & 上传图片成功后,选中图片并修改信息后提交 \\
             &之后点击右下角红色按钮:删除选中图片 \\

    \midrule
    输入数据 		& 任意图片 \\
    预期结果 		& 可以在相册内新建,修改,删除图片 \\
    实际结果             & 与预期结果完全一致 \\
    \bottomrule
  \end{tabular}
\end{table}


\subsection{教师,学生的邀请,转班,删除}

因为教师和学生管理界面和实现几乎相同, 所以下面统称人员。


\begin{table}[H]
  \centering
  \caption{人员的邀请}
  \label{tab:7}
  \begin{tabular}{cc}
    \toprule
    测试名称 & 人员的邀请  \\

    %%%%%%%%%%%%%%%%%%%%%%%%%%%%%%%%%
    测试人 & 张鑫语 \\
    测试日期 & 2015-05-11 \\

    \midrule
   \multirow{3}{*}{测试步骤} 		& 点击相关学校,在右边班级列表中,点击测试班级 \\
    & 再点击右边Panel中的教师管理, 之后点击右边Panel的邀请新教师 \\
      & 然后输入教师电话 \\
    \midrule
    输入数据 		& 11位数字 \\
    预期结果 		& 教师进入该班级 \\
    实际结果             & 与预期结果完全一致 \\
    \bottomrule
  \end{tabular}
\end{table}

\begin{table}[H]
  \centering
  \caption{人员的转班}
  \label{tab:8}
  \begin{tabular}{cc}
    \toprule
    测试名称 & 人员的转班  \\

    %%%%%%%%%%%%%%%%%%%%%%%%%%%%%%%%%
    测试人 & 张鑫语 \\
    测试日期 & 2015-05-11 \\

    \midrule
    \multirow{3}{*}{测试步骤} 		& 点击相关学校,在右边班级列表中,点击测试班级 \\
             & 再点击右边Panel中的教师管理,在教师列表中选择一个教师\\
             &  点击其右边的 Exchange Icon按钮, 然后选择转向的班级 \\

    \midrule
    输入数据 		& 班级的选择 \\
    预期结果 		& 教师从本班级移入目标班级 \\
    实际结果             & 与预期结果完全一致 \\
    \bottomrule
  \end{tabular}
\end{table}



\begin{table}[H]
  \centering
  \caption{人员的删除}
  \label{tab:9}
  \begin{tabular}{cc}
    \toprule
    测试名称 & 人员的删除  \\

    %%%%%%%%%%%%%%%%%%%%%%%%%%%%%%%%%
    测试人 & 张鑫语 \\
    测试日期 & 2015-05-11 \\

    \midrule
   \multirow{3}{*}{测试步骤} 		& 点击相关学校,在右边班级列表中,点击测试班级 \\
    & 再点击右边Panel中的教师管理,在教师列表中选择一个教师 \\
    & 点击其右边的垃圾箱按钮, 然后点击确定 \\

    \midrule
    输入数据 		& 人员的选择 \\
    预期结果 		& 教师从本班级中移走 \\
    实际结果             & 与预期结果完全一致 \\
    \bottomrule
  \end{tabular}
\end{table}







\subsection{帮手的邀请,修改,删除}


\begin{table}[H]
  \centering
  \caption{帮手邀请}
  \label{tab:11}
  \begin{tabular}{cc}
    \toprule
    测试名称 & 帮手邀请  \\

    %%%%%%%%%%%%%%%%%%%%%%%%%%%%%%%%%
    测试人 & 张鑫语 \\
    测试日期 & 2015-05-11 \\

    \midrule
    \multirow{2}{*}{测试步骤} 		& 点击最左边导航栏中的 “我的帮手”,  \\
                                        & 之后在主界面上部输入电话号码,之后点击 “发出邀请” \\
    \midrule
    输入数据 		& 11位数字 \\
    预期结果 		& 下面帮手列表添加了新的帮手 \\
    实际结果             & 与预期结果完全一致 \\
    \bottomrule
  \end{tabular}
\end{table}


\begin{table}[H]
  \centering
  \caption{帮手修改和删除}
  \label{tab:12}
  \begin{tabular}{cc}
    \toprule
    测试名称 & 帮手修改和删除  \\

    %%%%%%%%%%%%%%%%%%%%%%%%%%%%%%%%%
    测试人 & 张鑫语 \\
    测试日期 & 2015-05-11 \\

    \midrule
    \multirow{3}{*}{测试步骤} 		& 点击最左边导航栏中的 “我的帮手”, \\
             & 之后在主界面中部的帮手列表中, 修改帮手权限 \\
             & 之后点击提交, 然后点击右上角红色垃圾箱按钮。 \\
    \midrule
    输入数据 		& 权限的选择 \\
    预期结果 		& 提交后显示帮手权限被修改,删除后该帮手不出现在列表中 \\
    实际结果             & 与预期结果完全一致 \\
    \bottomrule
  \end{tabular}
\end{table}




\section{系统界面测试}



需要测试的系统界面:

\subsection{电脑端}

\begin{itemize}
\item 学校修改界面
\item 班级界面
\item 班级列表界面
\item 网盘界面
\item 相册界面
\item 教师,学生的管理界面
\item 帮手界面
\end{itemize}

\subsection{手机端}

\begin{itemize}
\item 发布状态界面
\item 浏览状态界面
\item 聊天界面
\item 我的班级和学校界面
\end{itemize}


\subsection{测试结果}

经由第五章的系统界面设计截图, 可以得知系统界面符合设计要求, 操作流畅, 为底层功能提供了比较好的UI接口。 具体截图不在这里冗余。
